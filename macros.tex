\newcommand{\name}{systemName}
% \renewcommand{\name}{main\xspace}

% ==============================================================================
% Inline comments
% ==============================================================================

\newcommand{\notes}[1]{}
\renewcommand{\notes}[1]{#1} % Comment this to disable comments
\newcommand{\authA}[1]{\notes{\textcolor{red}{AA: {#1}\xspace{}}}}
\newcommand{\authB}[1]{\notes{\textcolor{Green}{AB: {#1}\xspace{}}}}
\newcommand{\authC}[1]{\notes{\textcolor{purple}{AC: {#1}\xspace{}}}}
\newcommand{\authD}[1]{\notes{\textcolor{violet}{AD: {#1}\xspace{}}}}
\newcommand{\authE}[1]{\notes{\textcolor{Magenta}{AE: {#1}\xspace{}}}}

\newcommand{\todo}[1]{\notes{\textcolor{Aquamarine}{TODO: {#1}\xspace{}}}}
\newcommand{\help}[1]{\notes{\textcolor{Aquamarine}{HELP: {#1}\xspace{}}}}
\newcommand{\goal}[1]{\notes{\textcolor{blue}{GOAL: {#1}\xspace{}}}}

\newcommand{\cut}[1]{} % Content that can be cut
% \renewcommand{\cut}[1]{\textcolor{gray}{#1}} % Comment this line to apply the cut.

% ==============================================================================
% Wrappers
% ==============================================================================

\newcommand{\heading}[1]{\smallskip\noindent\textbf{#1.}}
\newcommand{\subheading}[1]{\underline{\emph{#1.}}}

% \titleformat*{\subparagraph}{\itshape}
\newcommand{\myparagraph}[1]{\paragraph{#1.}}
\newcommand{\mysubparagraph}[1]{\subparagraph{#1.}}

\newcommand{\eg}{e.g.\xspace}
\newcommand{\Eg}{E.g.\xspace}
\newcommand{\etc}{etc.\@\xspace}
\newcommand{\cf}{{cf.}\xspace}
\newcommand{\ie}{i.e.\xspace}
\newcommand{\Ie}{I.e.\xspace}
\newcommand{\etal}{et al.\xspace}

\newenvironment{packeditemize}{
\begin{itemize}[topsep=0pt, noitemsep, leftmargin=*]}
{\end{itemize}}

\newenvironment{packedenumerate}[1][(\arabic{*})]{
\begin{enumerate}[topsep=0pt, noitemsep, leftmargin=*, label=#1]}
{\end{enumerate}}

% \newcommand\vertarrowbox[3][6ex]{%
%   \begin{array}[t]{@{}c@{}} #2 \\
%   \left\uparrow\vcenter{\hrule height #1}\right.\kern-\nulldelimiterspace\\
%   \makebox[0pt]{\scriptsize#3}
%   \end{array}%
% }

% Circledsteps
% https://mirror.las.iastate.edu/tex-archive/macros/latex/contrib/circledsteps/circledsteps-manual.pdf
% \pgfkeys{/csteps/inner ysep=4pt}
% \pgfkeys{/csteps/inner color=white}
% \pgfkeys{/csteps/fill color=black}

% ==============================================================================
% Theorems, lemmas, and definitions
% ==============================================================================
% https://www.overleaf.com/learn/latex/Theorems_and_proofs
% https://gejian.wordpress.com/2010/02/16/autoref-in-latex/
% https://11011110.github.io/blog/2018/05/25/lipics-autoref-lemma.html
\newtheoremstyle{mytheoremstyle} % name
    {\topsep}                    % Space above
    {\topsep}                    % Space below
    {\itshape}                   % Body font
    {}                           % Indent amount
    {\scshape}                   % Theorem head font
    {.}                          % Punctuation after theorem head
    {.5em}                       % Space after theorem head
    {}  % Theorem head spec (can be left empty, meaning ‘normal’)
\theoremstyle{mytheoremstyle}
\newtheorem{theorem}{Theorem}[section]
\newtheorem*{theorem*}{Theorem}  % To have non-numbered theorems
\newtheorem{corollary}{Corollary}[theorem] % not needed for SIGCOMM template

\newtheorem{lemma}[theorem]{Lemma} % not needed for SIGCOMM template
% \newtheorem{lemma}{Lemma}[section] % If we want to have lemmas numbered separately from theorems
\newcommand{\lemmaautorefname}{Lemma}

\theoremstyle{mytheoremstyle}
\newtheorem{definition}{Definition}[section]

\theoremstyle{plain}
\newtheorem{conjecture}{Conjecture}[section]

\theoremstyle{remark}
\newtheorem*{remark}{Remark}

\renewcommand{\equationautorefname}{Eq.}
\renewcommand{\figureautorefname}{Fig.}

% ==============================================================================
% Highlighting and tables
% ==============================================================================

% % Highlighting
% \DeclareRobustCommand{\hlviolet}[1]{{\sethlcolor{violet!15}\hl{#1}}}
% \DeclareRobustCommand{\hlred}[1]{{\sethlcolor{red!15}\hl{#1}}}
% \DeclareRobustCommand{\hlgreen}[1]{{\sethlcolor{ForestGreen!20}\hl{#1}}}
% \DeclareRobustCommand{\hlblue}[1]{{\sethlcolor{blue!15}\hl{#1}}}

% % TabularX
% https://texdoc.org/serve/tabularx/0
% \def\tabularxcolumn#1{m{#1}}  % convert column from p to m.
% \newcolumntype{Y}{>{\centering\arraybackslash}X}
% \newcolumntype{Y}{>{\small\raggedright\arraybackslash}X}
% \newcolumntype{Y}{>{\hsize=.5\hsize\linewidth=\hsize}X}
% \newcolumntype{Y}{>{\hsize=1.5\hsize\linewidth=\hsize}X}

% ==============================================================================
% Colors
% ==============================================================================
\definecolor{TableGray}{rgb}{0.88,0.88,0.88}